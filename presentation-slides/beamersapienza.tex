\documentclass{beamer}
\usepackage{amsfonts,amsmath,oldgerm}
\usepackage{minted}
% \usepackage{minipage}
\usetheme{sintef}
\usepackage{setspace}

\usepackage{pgfpages}
%\setbeamertemplate{note page}[plain]
\setbeameroption{show notes on second screen=right}

\newcommand{\testcolor}[1]{\colorbox{#1}{\textcolor{#1}{test}}~\texttt{#1}}
\newcommand{\Pause}{\pause\note{\\.\hrulefill{}.\\}}

\usefonttheme[onlymath]{serif}

\titlebackground*{assets/background}

\newcommand{\hrefcol}[2]{\textcolor{cyan}{\href{#1}{#2}}}

\title{Automazione dell'analisi statica e dinamica di malware}
\subtitle{}
\course{Laurea in Informatica}
\author{\href{mailto:sestito.1937764@studenti.uniroma1.it}{Simone Sestito}}
\IDnumber{1937764}
\date{Anno Accademico 2022/2023}
 
\begin{document}
\maketitle
\note{
Il progetto si focalizza sull'\textbf{automazione} di analisi di malware. \\
Realizzato nella fattispecie di un tirocinio svolto \textbf{in azienda}, e dando come prodotto uno strumento:
%
\begin{itemize}
    \item Integrabile con il resto dei sistemi aziendali
    \item Facilmente scalabile ed estensibile, in ottica futura
\end{itemize}
}

\chapter{Introduzione}

\section{Malware e tipologie}
Nell'ambito della sicurezza informatica, un malware è definito come un software che attua comportamenti malevoli contrari alla volontà dell'utente finale con lo scopo di modificare il normale funzionamento di un sistema, a illegittimo vantaggio del suo autore e ai danni dell'utente \cite{malwarebytes_malware_definition}.

Possono essere di varie tipologie, o unioni di alcune di esse:
\begin{itemize}
    \item \textbf{Adware}: progettati per presentare messaggi pubblicitari e guadagnare da essi
    \item \textbf{Spyware}: hanno lo scopo di osservare le azioni dell'utente senza il suo permesso per poi riportare il tutto all'autore
    \item \textbf{Virus}: vanno ad alterare altri programmi agganciandoci del proprio codice
    \item \textbf{Worm}: alterano altri computer in una stessa rete, provocando danni
    \item \textbf{Trojan}: ingannano l'utente presentandosi come un software utile, che quindi l'utente va ad installare volontariamente, non sapendo cosa si cela realmente al contrario di ciò che gli è stato promesso
    \item \textbf{Ransomware}: guadagnano sul malcapitato criptando tutti i propri file importanti, chiedendo poi un riscatto in criptovalute per lo sblocco - qui si fa leva sull'importanza e l'assenza di backup di dati importanti e sul proprio valore sia economico che morale (a seconda che si tratti di un'azienda o di un utente domestico)
    \item \textbf{Rootkit}: sfruttano vulnerabilità nel sistema target per ottenere privilegi da amministratore
\end{itemize}
Com'è normale pensare, non esistono solo questi tipi, ma sono solo alcuni dei più famosi.

Infine, per maggiore comprensione delle seguenti illustrazioni, è bene spiegare altri pochi termini:
\begin{itemize}
    \item Una \textbf{vulnerabilità} è rappresentata da una criticità in un componente di un sistema informatico, come l'assenza di misure di sicurezza o la propria compromissione
    \item Un \textbf{exploit} invece è un software o un insieme di comandi che vanno a sfruttare la vulnerabilità a proprio vantaggio, al fine di provocare un comportamento altrimenti inaspettato
    \item Un tipo particolare di exploit chiamato \textbf{zero-day} va a sfruttare falle non note prima dell'attacco, provocando così potenzialmente molti più danni, non essendo ancora disponibile una patch del software vulnerabile
    \item Per \textbf{IoC} si intende un Indicator of Compromise, indicatore usato per identificare indirizzi IP o nomi di dominio di C\&C, hash di file, firme di antivirus, o altro, che faccia ricondurre con alta probabilità a una specifica intrusione
\end{itemize}

\subsection{MITRE ATT\&CK}
\label{chap:mitre_attack}

\begin{figure}[htbp]
    \centering
    \includegraphics[width=\textwidth]{assets/mitre_attack_matrix.png}
    \caption{MITRE ATT\&CK Matrix for Enterprise}
    \label{fig:mitre_attack_matrix}
\end{figure}

Per identificare più precisamente le tipologie di malware che stiamo trattando, si va ad usare il \emph{MITRE ATT\&CK® Framework} (Adversarial Tactics, Techniques, and Common Knowledge)
\footnote{Pagina ufficiale MITRE ATT\&CK: \url{https://attack.mitre.org/}}.

Si tratta di una base di conoscenza sviluppata da MITRE Corporation, che include tattiche e adversary techniques, basata osservando gli avvenimenti nel mondo reale. Nato nel 2013 con lo scopo di descrivere le più comuni tattiche, tecniche e procedure \emph{(TTP)} usate nei sistemi Windows enterprise, ad uso interno MITRE, è poi diventato lo standard de-facto per la descrizione di tali aspetti di un attacco
\cite{mitre_attack_framework_introduction}, per far fronte all'assenza di una tassonomia comune per la descrizione dei TTP.

Si focalizza sulle interazioni che il malware ha col sistema target, e raggruppa questi comportamenti in tattiche per fornire più contesto sulla tecnica utilizzata.
\begin{itemize}
    \item Le \textbf{tattiche} sono il \emph{perché} di una tecnica, indicando l'obiettivo che si intende perseguire, e servono da contenitore per le varie tecniche; nello standard iniziano per \texttt{TA} (es: \texttt{TA0043} - Reconnaissance - Ricognizione, l'avversario va ad ottenere informazioni sul sistema per capire come muoversi)
    \item Le \textbf{tecniche} invece sono il \emph{come} di una specifica azione, e potrebbero far dedurre anche il \emph{cosa} un avversario ottiene come risultato della sua azione - particolarmente utile nella tattica TA0007 Discovery, dove si cerca di studiare la composizione del sistema target. \\
    Una tecnica è spesso composta da sotto-tecniche per essere ancora più specifici.
    Ad esempio, la tecnica \texttt{T1595.002} Vulnerability Scanning (dove si va capendo i software disponibili sul sistema target e la loro versione, con il possibile scopo di verificare se si allinea a una specifica versione vulnerabile di cui l'avversario già dispone di un exploit) è sotto-tecnica di \texttt{T1595} Active Scanning, e fa parte della tattica \texttt{TA0043} Reconnaissance vista al punto precedente.
\end{itemize}

Nella matrice in figura \ref{fig:mitre_attack_matrix}, le tattiche sono le colonne e le tecniche sono le celle, possibilmente composte da sotto-tecniche.

Viene usato anche per l'integrazione con la Cyber Threat Intelligence, che vedremo nella sezione \ref{chap:cyber_threat_intelligence}, punto focale anche degli strumenti realizzati nel progetto.

\section{Cyber Kill Chain}
Partendo dalle tattiche appena viste, possiamo costruire ciò che è noto col nome di \emph{Cyber Kill Chain}.
Si tratta di un modello simile al MITRE ATT\&CK framework (sez. \ref{chap:mitre_attack}), ma segue un diverso approccio. Qui infatti si vanno a ripercorrere le 7 tipiche fasi cronologiche di un attacco ad un sistema informatico, fornendo una panoramica più ad alto livello.

\begin{figure}[h]
    \centering
    \includegraphics[width=0.6\textwidth]{assets/cyber_kill_chain.png}
    \caption{(diritti di uso dell'immagine concessi dall'autore "Fondazione F3RM1")}
    \label{fig:cyber_kill_chain}
\end{figure}

Si compone delle seguenti fasi \cite{cyber_kill_chain_360}, portando un esempio noto, alla base anche della challenge \emph{Pilgrimage} della piattaforma \emph{HackTheBox}:
\begin{enumerate}
    \item \textbf{Ricognizione}: identificazione dei punti di accesso a un sistema e sue vulnerabilità, derivabili tramite analisi attiva (uso di strumenti come \emph{nmap} in maniera più o meno aggressiva) o passiva, leggendo informazioni già da fonti note (come \emph{Shodan.io} partendo da un indirizzo IP): nell'esempio notiamo che è esposta la cartella .git della Git repository, da cui estraiamo il sorgente della webapp, trovando l'uso di ImageMagick 7.1.0-49
    
    \item \textbf{Armamento}: creazione del vero e proprio malware da utilizzare nell'attacco, nell'intento di sfruttare la vulnerabilità individuata - nell'esempio creiamo la nostra immagine PNG ad-hoc

    \item \textbf{Consegna}: il payload creato deve essere consegnato alla vittima in qualche modo, come una mail di phishing o un form di una pagina del sito web - quindi eseguiamo l'upload del PNG nel form dal sito

    \item \textbf{Exploit}: nella realtà dei fatti, si sta sfruttando una vulnerabilità individuata - nel nostro caso reale, sfruttiamo la CVE-2022-44268 e leggiamo il file del database della webapp, incluse le password

    \item \textbf{Installazione}: dopo la riuscita dell'exploit, si scarica e avvia il payload malevolo, cercando di bypassare strategie esistenti di rilevazione, usando metodologie come obfuscation - nell'esempio, accediamo via SSH con la password e possiamo installare il nostro sistema

    \item \textbf{Command and Control (C\&C)}: andiamo a instaurare un canale di comunicazione tra noi e la vittima, così da controllarlo da remoto - la forma più basilare è l'avvio di una Reverse Shell criptata

    \item \textbf{Azioni sugli obiettivi}: eseguiamo le azioni più disparate, in base a ciò che vogliamo fare noi come attaccanti - nel caso CTF, andremo a leggere un file contenente il flag, simbolicamente un dato sensibile sul sistema della vittima
\end{enumerate}

\section{Tipi di sistemi di protezione}
Ora che abbiamo visto tutte queste minacce, viene spontaneo chiedersi come sia possibile proteggersi.

Il tipo più basilare di software per la protezione è l'\textbf{antivirus}. Questa tipologia di strumenti di protezione si va a concentrare sui singoli computer endpoint, e sfrutta modelli euristici per la rilevazione di vari tipi noti di malware, analizzando \emph{file su disco}.

Ben più potente è l'EDR, acronimo di \textbf{Endpoint Detection and Response}, che sfrutta l'\emph{analisi comportamentale} per rilevare e contrastare anche minacce sconosciute, basandosi sui loro comportamenti. Ad esempio, se vediamo che, dopo aver aperto un file scaricato da Internet, il processo del programma che ne permette la visualizzazione inizia a fare operazioni fuori dal normale funzionamento, come aprire una shell o modificare file o registri di sistema, sicuramente andrà terminato e riportato come incidente, che poi andrà gestito da chi di competenza.
La sostanziale differenza è il passaggio da semplice analisi su disco a una profonda analisi comportamentale, sfruttando strumenti forniti dal sistema operativo come Microsoft ETW \emph{(Event Tracing for Windows)} o eseguire l'hooking delle syscall manualmente - tecnica ben più complessa ed error-prone.

Infine, con un XDR (\textbf{eXtended Detection and Response}) andiamo ad allargare gli elementi sotto la protezione del software in questione, permettendo un'integrazione anche con reti, firewall, log di sicurezza e molto altro.
In questo modo, ha la potenza di aggregare e correlare eventi da più fonti, rilevando minacce sofisticate che verrebbero altrimenti ignorate.
Un esempio in questa categoria è \emph{SentinelOne XDR}.

\section{Cyber Threat Intelligence}
\label{chap:cyber_threat_intelligence}
L'area di Threat Intelligence si occupa di raccogliere tutta una serie di dati dai vari elementi,
per poi aggregarli e compiere analisi sulle informazioni ottenute, al fine di rilevare minacce anche sofisticate e prendere decisioni sulla sicurezza in maniera più veloce e consapevole.

La Threat Intelligence è un tassello estremamente importante nella sicurezza di un sistema informatico, soprattutto se unito al MITRE ATT\&CK framework: documentando i profili con cui l'avversario opera (come \texttt{APT29} - i profili sono insiemi di attività tipiche del modo di operare di alcuni gruppi nel panorama cyber), ma anche individuando la tecnica più efficacemente così da raggruppare tra loro più attacchi con caratteristiche comuni.
In questo modo, è possibile focalizzarsi maggiormente sulle tecniche che sono maggiormente usate in un dato profilo di attacco.

Esempi di come particolari avversari, delineati sotto un profilo, usano le proprie tecniche sono documentati nella pagina ATT\&CK corrispondente.
Studiandone la metodologia di attacco, è possibile replicarlo in fase di emulazione \emph{(Adversary Emulation)} per capire fino in fondo come difendersi da attacchi similari, e testare le proprie difese, nonché eseguire test di rilevazione.

Per questo e altri motivi, come vedremo, nella prima fase di analisi integreremo questo framework per la categorizzazione delle capabilities di un eseguibile.

\medskip

Un altro importante ruolo della Threat Intelligence, maggiormente vicino al progetto realizzato, è quello di analizzare i malware coinvolti negli attacchi noti o forniti da clienti qualora ne chiedessero l'analisi approfondita. Proprio a questo scopo, il proprio lavoro viene nettamente agevolato da strumenti di analisi automatica più estendibili possibile, così da avere fin da subito tutti i dati su cui lavorare.

\section{Contesto d'uso del progetto}
Una volta vista una panoramica superficiale di questo ambito della sicurezza informatica,
andiamo ad analizzare il contesto in cui si va ad inserire il progetto realizzato.

L'azienda ospitante offre, come suo core business, sistemi di EDR e XDR.
Questi sistemi usano programmi Agent installati nei vari endpoint, che sonde usate per analizzare il traffico di rete nei vari livelli ISO-OSI, nonché analisi sui dispositivi OT.

Soprattutto in riferimento all'Agent installato sugli endpoint, esso andrà a rilevare eseguibili con comportamenti anomali o sconosciuti e li andrà ad inviare al sistema di Threat Intelligence, che valuterà con una data confidence se siano o meno malevoli.
La capacità di eseguire detection di alta qualità è uno dei tratti distintivi tra le diverse soluzioni di sicurezza esistenti sul mercato e ha una diretta ripercussione sulla soddisfazione del cliente finale dell'azienda, oltre alla propria reputazione come brand.
Proprio in questo punto cruciale, viene l'esigenza dietro il progetto realizzato.

\subsection{Il problema da risolvere}
Lo scopo principale è la costruzione di un proprio sistema interno di analisi di malware automatizzata, eliminando le ultime dipendenze con servizi terzi quali VirusTotal, sia per un fattore economico che di flessibilità nei propri prodotti.

Un aspetto importante mantenuto chiaro per tutto il corso del progetto è proprio la flessibilità che deve essere garantita a questo nuovo strumento di essere esteso e maggiormente integrato nel corso della propria esistenza. Com'è intuibile infatti, il sistema qui creato verrà successivamente usato come base per la propria architettura di analisi malware automatizzata, che si andrà ad interfacciare con i vari prodotti.

\subsection{Architettura generale}

%%%%%%%%%%%%%%%%%%%%%%%%%%%%%%
%Agent? FSL? Chi parla con chi? Perchè? Quando?
%IN BREVISSIMO
%%%%%%%%%%%%%%%%%%%%%%%%%%%%%%
\section{Analisi statica}

\begin{frame}{Estrazione delle capabilities}
Dato un file eseguibile, andiamo a estrarre le \emph{capabilities}: quali azioni il programma è in grado di compiere

\vfill

\textbf{Come?} \\
Usiamo \emph{capa}, strumento open source realizzato da Mandiant

\vfill

\textbf{Perchè?} \\
Molto flessibile, con la possibilità di scrivere proprie regole custom in YAML
\end{frame}

\begin{frame}{Capabilities ottenute}
\begin{figure}
    \centering
    \includegraphics[width=\textwidth]{images/capa_example_invocation.png}
    \caption{Output restituito da capa}
    \label{fig:capa_example_invocation}
\end{figure}
\end{frame}

\begin{frame}{Parsing dell'output}
Non è un formato utilizzabile, bisogna farne il parsing.

\vfill

Notiamo la divisione in tabelle, e andiamo a trasformarlo in un JSON.

Lavoreremo sempre con formato JSON per essere sia human-readable dall'analista che machine-readable da servizi esterni.
\end{frame}

\begin{frame}{Output JSON}
\begin{figure}
    \centering
    \includegraphics[width=0.57\textwidth]{images/capa_json_output.png}
    \caption{Output dopo il parsing}
    \label{fig:capa_json_output}
\end{figure}
\end{frame}

\begin{frame}{Casi non supportati}
Viene fatta analisi statica del codice per estrarre le capabilities.

Questo non funziona nei casi di eseguibile pacchettizzato, installer o con runtime particolari (es: Visual Basic).
\vfill
capa terminerà con \textbf{exit code 14}
\end{frame}

\begin{frame}{Conseguenze del crash}
\begin{itemize}
    \item Può portare a un fallimento dell'intero processo di analisi statica
    \item Crea spreco di risorse (CPU e memoria) per diversi minuti
\end{itemize}
\end{frame}

\begin{frame}{Rilevazione del caso non supportato}
Ci sono regole capa specifiche che rilevano questi casi, ma eseguirle richiede svariati minuti.

Dobbiamo creare un rilevatore custom da zero, basandoci sulle regole.

\begin{figure}
    \centering
    \includegraphics[width=\textwidth]{images/capa_detector_rule.png}
    \caption{Esempio di codice Python per simulare una regola}
    \label{fig:capa_detector_rule}
\end{figure}
\end{frame}

\begin{frame}{Struttura delle classi per il programma di rilevazione}
Le regole lavorano principalmente sul nome delle sezioni o presenza di stringhe.

\begin{figure}
    \centering
    \includegraphics[width=0.7\textwidth]{images/base_custom_static_analyzer.png}
    \caption{Diagramma delle superclassi}
    \label{fig:base_custom_static_analyzer}
\end{figure}
\end{frame}

\begin{frame}{Velocità del rilevatore}
La rilevazione durerà meno di un secondo, a fronte di svariati minuti iniziali.

\begin{figure}
    \centering
    \includegraphics[width=\textwidth]{images/custom_file_detector_output.png}
    \caption{Esecuzione del rilevatore}
    \label{fig:custom_file_detector_output}
\end{figure}
\end{frame}

\begin{frame}{YARA}
Si esegue anche signature-based matching con lo strumento YARA.
\vfill
Colleziona regole da varie realtà della cybersecurity (aziende, ricercatori, ...) \\
oltre a tutte quelle sviluppate internamente
\vfill
Per ottimizzare l'esecuzione, le regole sono state divise in cartelle ed eseguite solo quando necessario, in base al sistema operativo target
\end{frame}

\begin{frame}{Ulteriori strumenti}
\begin{itemize}
    \item \textbf{SSDeep} per fuzzy hashing e correlazione di sample simili tra loro
    \item \textbf{Detect-It-Easy} e \textbf{ExifTool} estraggono ulteriori metadati dal file in analisi (linker utilizzato, descrizioni, charset, tabella dei simboli, ...)
\end{itemize}

\vfill

Tutti integrati in un \emph{unico flusso di esecuzione} e uniti in uno stesso risultato JSON finale, continuando ad eseguire parsing e trasformazioni di ciò che otteniamo dagli strumenti grezzi
\end{frame}

\begin{frame}{Flusso di esecuzione}
\begin{figure}
    \centering
    \includegraphics[width=0.9\textwidth]{images/static_run_analysis_internal_tools.png}
\end{figure}
La gestione degli errori avviene tramite handle degli exit-code e restituzione di un JSON contenente il campo \emph{"error"} con un error code tra le stringhe predeterminate
\end{frame}

\begin{frame}{Containerizzazione}
Viene incluso in un container Docker comprensivo di tutte le dipendenze, per massimizzare la portabilità.

Grazie all'uso intensivo di \textbf{multi-stage build} si è ridotta la sua dimensione finale a $\approx$ 200 MB, partendo da oltre 2 GB se creata in modo naive

Sfrutta anche la cache per le build seguenti e la parallelizzazione usando \texttt{docker buildx}
\end{frame}

\begin{frame}{Multi-stage build}

\begin{figure}
    \centering
    \includegraphics[width=0.6\textwidth]{images/dockerfile.png}
\end{figure}
\end{frame}

\begin{frame}{Deploy su AWS}
Dopo un'analisi dei costi e delle caratteristiche di AWS Lambda, EC2 e simili, si è optato per eseguire questa analisi su AWS Lambda.
\vfill
Si creano varie Lambda per minimizzare i permessi assegnati dove viene effettivamente analizzato il malware, seguendo il principio del \emph{Least Privilege}
\end{frame}

\begin{frame}{Architettura AWS}
\begin{figure}
    \centering
    \includegraphics[width=\textwidth]{images/aws_static_lambdas_architecture.png}
\end{figure}
\end{frame}

\begin{frame}{Compressione del risultato}
Per non usare inutilmente un bucket S3 per salvare i risultati, viene passato il JSON finale da \texttt{static\_analysis\_malware\_tools} a \texttt{static\_analysis\_reporting}.
\vfill
AWS limita questo payload a 6MB, spesso viene superato.
\vfill
Si è scelto di comprimere il JSON con gzip, poi reso una stringa ASCII usando Base64.
Se ancora si superano i 6MB si emette un errore, ma è un caso raro o dovuto a regole YARA mal scritte.
\end{frame}

\begin{frame}{Compressione del risultato}
\begin{figure}
    \centering
    \includegraphics[width=0.5\textwidth]{images/static_analysis_results_size.png}
    \caption{Differenza tra la dimensione prima e dopo la compressione}
    \label{fig:static_analysis_results_size}
\end{figure}
\end{frame}

\begin{frame}{Pipeline CI/CD}
Per automatizzare anche il processo di deploy dello strumento, si è creata una pipeline CI/CD su GitLab che esegue in appositi Runner.
\vfill
Così, se si modifica una regola o esegue una modifica, al push del commit Git viene eseguito:
\begin{itemize}
    \item \textbf{build} dell'immagine Docker
    \item \textbf{test} usando dei file innocui creati per varie architetture, sistemi operativi e casistiche (tradizionali, packed, installer, ...)
    \item \textbf{deploy} dell'immagine su AWS ECR e sulla Lambda in caso di test positivi
\end{itemize}
\end{frame}
\chapter{Analisi dinamica}

\section{Cuckoo Sandbox}
Sandbox per fare le nostre analisi.
Come funziona?

\section{Multiscanner}
Strumento \cite{anacondacon_multiscanner}

Usiamo Cuckoo dentro Multiscanner per avere dei report migliori e future integrazioni con altri strumenti che Multiscanner offre.

\subsection{Hash fuzzing}
Sempre nella presentazione \cite{anacondacon_multiscanner} con \textbf{ssdeep} al minuto 16:00.


\backmatter
\end{document}
