\chapter{Analisi statica}

\section{Struttura binari}

\subsection{ELF}

\subsection{PE}

\subsection{Tipi di exe packed e altri}

\section{IoC: Indicator of Compromise}

\section{Capa}
Includere anche il test su Colab con le statistiche

\section{Yara rules}
Sia per statici che in memoria dei processi 

\section{Havoc framework}
Non sempre però questa tecnica è sufficiente. Infatti, analizziamo un caso specifico: \emph{Havoc Framework}.

https://embee-research.ghost.io/havoc-c2-static-detection-via-ntdll-api-hashes/

Con questo famoso strumento di Command-and-Control è possibile andare a creare un proprio eseguibile Windows con dimensioni ridotte, in grado di svolgere Code Injection all'interno di altri eseguibili standard di Windows, come \texttt{notepad.exe} o \texttt{calc.exe}, caricando una libreria DLL che si occuperà del resto. In questo modo, nell'elenco dei processi è visibile l'applicativo Windows in questione e non il nostro programma, che magari dopo l'avvio aveva insospettito l'utente più attento - ma privo di strumenti più sofisticati come gli agent.
