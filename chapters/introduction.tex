\chapter{Introduction}

\section{Interessante}
Molto \cite{aws_inforce_malware}

\section{Cyber Kill Chain}

\section{Tipologie di malware}
Tipi come RAT, Ransomware, ....

\section{Come funziona la sicurezza nelle aziende}
\begin{itemize}
    \item Kill Chain, IDS, IPS https://youtu.be/Hnu4KDQOdcM
    \item Ruolo degli Agent e altri sistemi
\end{itemize}

\section{Ruolo ricoperto dall'azienda}

\section{Il problema}
Gyala Products (Endpoint, Network Probes, Mobile Agent) need a capability to enable detection of both unauthorised and malicious behaviours.
The ability of perform high quality detection is one of the distinctive trait between different security solutions and has a direct repercussion to customer satisfaction and brand reputation.  
Malware Analysis should be considered an integrated, as part of product development, and continuous process that adds value in gyala products and makes them effective and distinctive on the market.

lo scopo è ridurre i costi di virustotal, che si usa tantissimo, e potenziare il set di strumenti a disposizione dell'analista, potenzialmente integrandoli tutti in uno con sistema di plugin